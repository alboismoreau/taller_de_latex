%%%%%%%%%%%%%%%%%%%%%%%%%%%%%%%%%%%%%%%%%%%%%%%%%%%%%%%%%%%%%%%
%
% Welcome to Overleaf --- just edit your LaTeX on the left,
% and we'll compile it for you on the right. If you open the
% 'Share' menu, you can invite other users to edit at the same
% time. See www.overleaf.com/learn for more info. Enjoy!
%
%%%%%%%%%%%%%%%%%%%%%%%%%%%%%%%%%%%%%%%%%%%%%%%%%%%%%%%%%%%%%%%
\documentclass{article}
\title{TP2}
\author{anita, delfi, delfi, juan}
\date{23:55}
\begin{document}
\maketitle


Acá te muestro cómo se hace para usar negritas, cursiva o subrayar: \\

\emph{Biblioteca} \\ \textbf{indefinido}, \\
y tal vez infinito, de galer´ıas \underline{hexagonales}.\\
\\ Ahora te muestro como hacer una lista de items:\\
\begin{itemize}
\item Mate
\item Café
\item Harina
\item Palmitos
\end{itemize}\\ Ahora te explico cómo hacer una ecación:
\begin{equation}
\alpha + \beta + 1
\end{equation} 
%esto de acá es un comentario, va a ignorar el resto de la línea

Las palabras se separan
por uno o más espacios.


Los párrafos por una o
mas líneas en blanco.


El universo se compone
de un n´umero(\ldots)


Hay algunos caracteres con significados especiales en LATEX. Si los escrib´ıs, te
va a dar un error. Ten´es que escaparlos con un \.
\$\%\&\#!

% :(
Sean a y b enteros positivos.
Sea c = a - b + 1.

% :)
Sean $a$ y $b$ enteros positivos.
Sea $c = a - b + 1$.

Se usan siempre de a pares - uno para comenzar el modo matem´atico y otro
para salir. LATEX maneja los espacios autom´aticamente.
Sea $y=mx+b$ \ldots
Sea $y = m x + b$ \ldots


Si tu ecuación es muy importante, podés mostrarla dentro del entorno
equation.
Las raices de una ecuación cuadrática son

\begin{equation}
x = \frac{-b \pm \sqrt{b^2 - 4ac}}
{2a}
\end{equation}

donde $a$, $b$ y $c$ son \ldots






\end{document}

