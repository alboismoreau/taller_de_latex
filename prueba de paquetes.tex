\documentclass{article}
\usepackage{listings}
\usepackage{algpseudocode}
\usepackage[utf8]{inputenc}
\usepackage[spanish]{babel}
\usepackage{amsmath}
\usepackage{graphicx}
\usepackage{algorithm}

\begin{document}
Un escenario común, es el querer incluir código en nuestro documento. Por ejemplo:

\begin{lstlisting}[language=c]
int sumar(int x, int y) {
return x + y;
}
\end{lstlisting}
Y para ello utilizamos el paquete \emph{listings}

\begin{figure}[h]
\centering
\includegraphics
{../images/lion.png}
\caption{Imagen de un leon}
\end{figure}


Por último, queremos poder tomar un paso atrás y escribir en abstracto un pesudocógido, más allá del lenguaje en el que lo implementemos.
Para ello, podemos utilizar el paquete \emph{algpseudocode} con el entorno \emph{algorithmic}. Y dentro del mismo, incorporamos el pseudocódigo de
nuestro programa.
Por ejemplo, si queremos escribir un condicional if -else, usamos

\begin{algorithmic}
\If{a}
   \State b
\Else
    \State c
\EndIf
\end{algorithmic}


O si queremos un ciclo de iteración
condicional for, usamos:

\begin{algorithmic}
\For{i e 1, . . . , n}
\State a ← a + i
\EndFor
\end{algorithmic}


Acá va un ejercicio del taller de latex
\begin{algorithm}
\caption{SUMARTUPLAS}

\begin{algorithmic}
    \Function{sumarTuplas}{tuplas}
        \State $suma \gets (0,0)$
        \If{$tuplas \neq \{\}$}
            \For{$i = 1$ to $|tuplas|$}
                \State $suma_1 \gets tuplas[i]_1$
                \State $suma_2 \gets tuplas[i]_2$
            \EndFor
        \EndIf
        \State \Return $suma$
    \EndFunction
\end{algorithmic}
\end{algorithm}



\end{document}




