\documentclass[12pt]{article}

% --------------------------------------------------
% Paquetes para idioma y codificación
% --------------------------------------------------
\usepackage[utf8]{inputenc}       % Permite usar acentos directamente
\usepackage[spanish]{babel}       % Traduce elementos automáticos al español

% --------------------------------------------------
% Márgenes de página
% --------------------------------------------------
\usepackage[margin=2.5cm]{geometry}   % Define márgenes de todo el documento

% --------------------------------------------------
% Paquetes para matemáticas y lógica
% --------------------------------------------------
\usepackage{amsmath, amssymb}     % Simbología matemática
\usepackage{mathtools}            % Extiende amsmath

% --------------------------------------------------
% Paquetes para código fuente
% --------------------------------------------------
\usepackage{listings}             % Para insertar código
\usepackage{xcolor}               % Para colorear sintaxis del código

% --------------------------------------------------
% Hipervínculos
% --------------------------------------------------
\usepackage[colorlinks=true, linkcolor=blue, urlcolor=blue]{hyperref} % Enlaces clicables

% --------------------------------------------------
% Datos del documento
% --------------------------------------------------
\title{Trabajo Práctico \#1\\\large Algoritmos y Estructuras de Datos}
\author{Nombre del estudiante}
\date{\today}

% --------------------------------------------------
% Configuración listings para mostrar código Java
% --------------------------------------------------
\lstdefinelanguage{JavaCustom}{
  language=Java,
  morekeywords={assert,requires,ensures,modifies},  % Palabras clave extra si se usan en especificaciones
  keywordstyle=\color{blue},                         % Color de palabras clave
  commentstyle=\color{gray},                         % Color de comentarios
  stringstyle=\color{red},                           % Color de cadenas
  basicstyle=\ttfamily\small,                       % Fuente tipo máquina de escribir
  numbers=left,                                       % Números de línea a la izquierda
  numberstyle=\tiny\color{gray},                    % Estilo de los números
  stepnumber=1,                                       % Mostrar cada línea
  numbersep=5pt,                                      % Separación entre código y número
  showstringspaces=false,                            % No marcar espacios en strings
  tabsize=2,                                          % Tamaño del tabulador
  breaklines=true                                    % Permitir saltos de línea automáticos
}

\lstset{language=JavaCustom} % Aplicar configuración a todos los listings

% --------------------------------------------------
% Comandos útiles para especificaciones formales
% --------------------------------------------------
\newcommand{\Pre}[1]{\textbf{Pre:}~#1}       % Precondición
\newcommand{\Post}[1]{\textbf{Post:}~#1}     % Postcondición
\newcommand{\Inv}[1]{\textbf{Inv:}~#1}       % Invariante de bucle (si aplica)
\newcommand{\Func}[2]{\textbf{Función auxiliar}~#1:~#2}  % Definición de función auxiliar

\begin{document}

\maketitle

\section{Enunciado}
Escribir un procedimiento que calcule el máximo común divisor de dos enteros positivos, usando el algoritmo de Euclides.

\section{Especificación}

% Explicamos qué queremos que haga el procedimiento usando lógica

Sean $a, b \in \mathbb{Z}^{+}$. Queremos especificar el procedimiento:

\begin{itemize}
  \item \Pre{$a > 0 \land b > 0$}  % Condición de entrada: ambos números deben ser positivos
  \item \Post{El resultado es $mcd(a, b)$ tal que $\forall d \in \mathbb{Z}^{+}, d \mid a \land d \mid b \Rightarrow d \mid \text{resultado}$}  % Condición de salida: el resultado es un divisor común máximo
\end{itemize}

% Ejemplo de función auxiliar declarada de forma formal
\Func{esDivisorComún}{\forall d: d \mid a \land d \mid b \Rightarrow d \mid x}

\section{Desarrollo}

% Breve explicación del algoritmo usado

Usamos el algoritmo de Euclides que consiste en aplicar divisiones sucesivas:

\begin{itemize}
  \item Mientras $b \neq 0$, asignar $a \leftarrow b$ y $b \leftarrow a \bmod b$
  \item Cuando $b = 0$, el MCD es $a$
\end{itemize}

\section{Código Java}

% Código Java correspondiente a lo descrito

\begin{lstlisting}
public class MCD {

    public static int mcd(int a, int b) {
        while (b != 0) {
            int temp = b;
            b = a % b;
            a = temp;
        }
        return a;
    }

}
\end{lstlisting}

\end{document}
